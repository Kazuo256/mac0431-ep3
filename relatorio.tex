\documentclass[a4paper,11pt]{article}
\usepackage[T1]{fontenc}
\usepackage[utf8]{inputenc}
\usepackage{lmodern}
\usepackage{hyperref}

\title{MAC0431 - Segundo Semestre de 2012 \\
       EP3 - Relatório}
\author{Thiago de Gouveia Nunes - Nº USP: 6797289 \\
        Wilson Kazuo Mizutani - Nº USP: 6797230}

\begin{document}

\maketitle

\section{Organização dos Arquivos}

  A pasta raiz do trabalho deve conter os seguintes arquivos:
  
  \hspace{50pt}
  \begin{itemize}
    \item \textbf{src/}:
      Pasta com o código fonte do programa.
    \item \textbf{Makefile.common}:
      Arquivo auxiliar de Makefile que veio com os exemplos da CGMLib.
    \item \textbf{relatorio.pdf}:
      Esse relatório.
    \item \textbf{input, input2 e input3}:
      Amostras de arquivos de entrada.
    \item \textbf{run.sh}:
      Script bash para rodar o programa.
  \end{itemize}

\section{Compilação}

  Para compilar nosso programa, é preciso entrar na pasta \verb$src/$ e usar o
  comando \verb$make$:

  \begin{verbatim} $ cd src/
 $ make \end{verbatim}

\section{Modo de Uso}

  Para usar nosso programa, deve-se rodar o script bash \verb$run.sh$ a partir da pasta
  raiz:
  
  \begin{verbatim} $ ./run.sh <n> <entrada> \end{verbatim}
  
  Onde:
  
  \begin{itemize}
    \item[<n>]
      Número de processadores que devem ser usados para resolver o problema.
    \item[<entrada>]
      Caminho para o arquivo de entrada, cujo formato é explicado adiante.
  \end{itemize}
 
\section{Formato da Entrada}

  O arquivo de entrada deve conter as dimensões das matrizes em sequência, com
  um zero indicando o final. Por exemplo, o arquivo \verb$input$ contém a
  sequência:

  \begin{verbatim}                    10 100 50 20 80 0 \end{verbatim}

  Indicando a multiplicação de 4 matrizes com dimensões 10x100, 100x50, 50x20 e
  20x80, nessa ordem.

\section{Detalhes de Implementação}

  Seguimos bem fielmente a ideia proposta no artigo\footnotemark. As únicas
  partes mais "abertas" da implementação é como lidar com casos em que o
  número \verb$p$ de processadores não divide o número \verb$n$ de matrizes e
  como estruturar os dados nas mensagens entre os processos.

  \footnotetext{
    \textit{"A Parallel Chain Matrix Product Algorithm on the InteGrade Grid"}
    indicado no enunciado.
  }

  Para que o programa lide com o caso em que \verb$p$ não divide \verb$n$, sem
  ter que ficar tratando casos especiais demasiadamente, usamos os valores
  \verb$n/p$ (divisão inteira) e \verb$n%p$, chamados de \verb$block_size$ e
  \verb$extra_size$. Seguimos a explicação do artigo e dividimos a matriz de
  custos em blocos de \verb$block_size$ campos de largura e altura cada, com a
  exceção dos blocos da última coluna e da última linha, que têm mais
  \verb$extra_size$ campos de largura e de altura, respectivamente.

  Na troca de mensagens, simplesmente usamos a classe \verb$BasicCommObject$
  com o tipo \verb$unsigned$ e serializamos cada bloco em uma
  \verb$CommObjectList$. Com os valores que cada processo dispõe é possível
  tanto serializar o bloco quanto recuperar as informações quando a mensagem
  é recebida. Isso é feito nas funções \verb$send_block$ e \verb$receive_block$,
  respectivamente.

  A função \verb$compute_block$ calcula os campos da matriz de custos
  correspondentes a cada turno do algoritmo. Para não ter que tratar o primeiro
  turno como um caso separado dos demais (pois há menos campos a serem
  calculados nele), essa função itera nas linha e colunas, de baixo pra cima e
  da esquerda para a direita, ao invés de pelas diagonais como no algritmo
  sequencial, pois (1) essa ordem que usamos garante que os campos são
  calculados em uma ordem válida e (2) basta somar um deslocamento nos índices
  das colunas quando se está no primeiro turno para lidar com ele.
  
\section{Dificuldades e Facilidades}

  A parte mais fácil foi implementar o esqueleto geral do algoritmo e as trocas
  de mensagens usando a serialização dos blocos. Até ter a primeira versão
  funcionando para só um processador foi tranquilo também. O problema começou
  quando aumentamos a entrada e tentamos com mais processadores. Não chegou a
  ser nada preocupante, apenas um monte de cuidados que precisaram ser tomados
  com os índices no algoritmo.

  Fora isso, também demoramos um pouco para nos adaptarmos ao trabalho na VM.
  Principalmente fazer a internet funcionar nela, para que pudéssemos usar SSH
  e Git. Depois disso ela deixou completamente de ser um empecilho ao trabalho.

  E entender os exemplos e a API da CGMLib foi até que relativamente simples,
  pelo menos no que diz respeito ao necessário para fazer o EP. O único detalhe
  um pouco mais estranho foi que precisamos usar uma chamada ao método
  \verb$synchronize$ para que as saídas dos processos (mesmo que direcionadas
  para arquivos separados) ficassem corretamente impressas.

\end{document}

